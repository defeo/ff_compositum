\documentclass[12pt]{article}

\usepackage{bbm,fullpage}
\usepackage{amsmath}
\usepackage{alltt, amssymb,stmaryrd,amsthm}

\usepackage{algorithm}
\usepackage{algorithmic}
\renewcommand{\algorithmicrequire}{\textbf{Input:}}
\renewcommand{\algorithmicensure}{\textbf{Output:}}

\def\C {\ensuremath{\mathsf{C}}}
\def\M {\ensuremath{\mathsf{M}}}
\def\Q {\ensuremath{\mathbb{Q}}}
\def\N {\ensuremath{\mathbb{N}}}
\def\R {\ensuremath{\mathbb{R}}}
\def\Z {\ensuremath{\mathbb{Z}}}
\def\F {\ensuremath{\mathbb{F}}}
\def\H {\ensuremath{\mathbb{H}}}
\def\K {\ensuremath{\mathbb{K}}}
\def\Kbar {\ensuremath{\overline{\mathbb{K}}}}
\def\L {\ensuremath{\mathbb{L}}}
\def\A {\ensuremath{\mathbb{A}}}

\def\mul {\ensuremath{\mathrm{mul}}}
\def\div {\ensuremath{\mathrm{div}}}
\def\rem {\ensuremath{\mathrm{rem}}}
\def\cat {\ensuremath{\mathrm{cat}}}
\def\coeff {\ensuremath{\mathrm{coefficient}}}
\def\mulmod {\ensuremath{\mathrm{mulmod}}}
\def\rev {\ensuremath{\mathrm{rev}}}
\def\x {\ensuremath{\mathbf{x}}}
\def\Tr {\ensuremath{\mathrm{Tr}}}

\newtheorem{Def}{Definition}
\newtheorem{Theo}{Theorem}
\newtheorem{Prop}{Proposition}
\newtheorem{Lemma}{Lemma}

\begin{document}

%%%%%%%%%%%%%%%%%%%%%%%%%%%%%%%%%%%%%%%%%%%%%%%%%%%%%%%%%%%%
%%%%%%%%%%%%%%%%%%%%%%%%%%%%%%%%%%%%%%%%%%%%%%%%%%%%%%%%%%%%
%%%%%%%%%%%%%%%%%%%%%%%%%%%%%%%%%%%%%%%%%%%%%%%%%%%%%%%%%%%%

In all that follows, $\K$ is a perfect field. We give algorithms 
for embedding and isomorphism.

%%%%%%%%%%%%%%%%%%%%%%%%%%%%%%%%%%%%%%%%%%%%%%%%%%%%%%%%%%%%
%%%%%%%%%%%%%%%%%%%%%%%%%%%%%%%%%%%%%%%%%%%%%%%%%%%%%%%%%%%%
%%%%%%%%%%%%%%%%%%%%%%%%%%%%%%%%%%%%%%%%%%%%%%%%%%%%%%%%%%%%

\section{Basic algorithms}

In this section, we review several known algorithms. We give
pseudo-code for those that cannot be found in standard references such
as~\cite{vzGG}, and that are not available in standard computer
algebra systems.

We denote by $\M:\N \to \N$ a function such that polynomials in
$\K[X]$ of degree at most $n$ can be multiplied in $\M(n)$ operations
in $\K$, and we make the usual super-linearity assumptions on
$\M$~\cite[Chapter~8]{vzGG}. We also denote by $\omega$ a constant in
$(2,3]$ such that one can multiply matrices of size $n$ over $\K$
using $O(n^\omega)$ operations in $\K$.

Given variables $X_1,\dots,X_s$ and integers $d_1,\dots,d_s$,
$\K[X_1,\dots,X_s]_{d_1,\dots,d_s}$ denotes the set of polynomials $P$
in $\K[X_1,\dots,X_s]$ such that $\deg(P,X_i) < d_i$ holds for all
$i$.

%%%%%%%%%%%%%%%%%%%%%%%%%%%%%%%%%%%%%%%%%%%%%%%%%%%%%%%%%%%%

\subsection{Polynomial multiplication and remainder}

For $B$ in $\K[X]$ of degree at most $m$, it will be convenient to let
$$
\begin{array}{cccc}
\mul(.,B,m,k): &\K[X]_k& \to &\K[X]_{k+m}\\
& A & \mapsto & AB
\end{array}$$ 
denote the multiplication-by-$B$ operator, which can be applied using
$\M(\max(m,k))$ operations in $\K$. In a similar vein, we denote by
$$
\begin{array}{cccc}
\rev(.,m): &\K[X]_m &\to& \K[X]_m  \\
& F & \mapsto & X^{m-1} F(1/X)
\end{array}$$ the reversal operator. 

Next, fix a monic polynomial $P$ of degree $m$ in $\K[X]$. For $k \ge 1$, we denote by
$\rem(.,P,k)$ the operator
$$
\begin{array}{cccc}
\rem(.,P,k): &\K[X]_k& \to &\K[X]_{m}\\
& A & \mapsto & A \bmod P.
\end{array}$$ 
Such remainders can be computed in time $O(\M(\max(k,m)))$ using the
Cook-Sieveking-Kung algorithm~\cite[Chapter~9]{vzGG}. For $B$
in $\K[X]/\langle P \rangle$ we will also use the modular multiplication
operator
$$\begin{array}{cccc} \mulmod(.,B,P): & \K[X]/\langle P \rangle & \to
  & \K[X]/\langle P \rangle\\ & A & \mapsto & AB \bmod P.
\end{array}$$ 
Identifying $\K[X]_m$ and $\K[X]/\langle P \rangle$, and using the
direct algorithm for modular multiplication (multiply, then reduce),
we can write $$\mulmod(.,B,P) = \rem(.,P,2m-1) \circ \mul(.,B,m-1,m);$$
one can thus compute $\mulmod(A,B,P)$ using $O(\M(m))$ operations in
$\K$, as is well known.

Finally, we say a brief word about bivariate algorithms.  Given two
integers $d,e$, we denote by
$$
\begin{array}{cccc}
\rev(.,d,e): &\K[X,Y]_{d,e} &\to& \K[X,Y]_{d,e}  \\
& F & \mapsto & X^d Y^e F(1/X,1/Y)
\end{array}$$
the operator that reverses a polynomial $F$ with respect to both $X$
and $Y$ (and does not require any arithmetic operation).

Given $P$ and $Q$ monic in respectively $\K[X]$ and $\K[Y]$, 
and given $B$ in $\K[X,Y]/\langle P,Q\rangle$, we will use the
bivariate modular multiplication operator
$$\begin{array}{cccc} \mulmod(.,B,\langle P,Q \rangle): & \K[X,Y]/\langle P,Q \rangle & \to
  & \K[X,Y]/\langle P,Q \rangle\\ & A & \mapsto & AB \bmod \langle P, Q \rangle.
\end{array}$$ 
Applying this operator take $O(\M(de))$ operations in $\K$ (todo: ref).


\paragraph{Transposed algorithms.}
Next, we briefly review the transposed version of some the operations
seen above. In what follows, the dual of a vector space such as
$\K[X]_m$ or $\K[X]/\langle P \rangle$, with $P$ of degree $m$, is
simply seen as $\K^m$.

The transpose of the multiplication-by-$B$ map is denoted by
$$\mul^t(.,B,m,k): \K^{k+m} \to \K^k;$$ to implement it, one can use
transposed versions of plain, Karatsuba and
FFT algorithms~\cite{bostan+lecerf+schost:tellegen,hanrot+quercia+zimmermann},
which have the same running time, up to an extra $O(m)$.
By identifying $\K[X]_k$ with its dual, one can also see
$\mul^t(.,B,m,k)$ mapping $\K[X]_{k+m}$ to $\K[X]_{k}$ and notice that
$\mul^t(.,B,m,k)$ then becomes $$A \in \K[X]_{k+m} \mapsto
(A\ \rev(B,m+1) \bmod X^{k+m}){\rm~div~}X^{m} \in \K[X]_k;$$ this
formula leads to algorithms for the transposed product that can be
implemented using only ``classical'' polynomial multiplication, but
are slower than those
of~\cite{bostan+lecerf+schost:tellegen,hanrot+quercia+zimmermann} by a
constant factor. 

The reversal operator is its own tranpose. Finally, taking $P$ monic
of degree $m$ in $\K[X]$, we discuss the tranposes of $\rem$ and
$\mulmod$. Since elements of $\K[X]/\langle P \rangle$ are given on
the monomial basis $1,X,\dots,X^{m-1}$, a linear form $\ell:
\K[X]/\langle P \rangle \to \K$ is represented by its values on the
monomial basis, that is, by the sequence $(\ell_i=\ell(X^i))_{0 \le i
  < m}$.

As showed in~\cite{bostan+lecerf+schost:tellegen}, the transposed map
$$
\begin{array}{cccc}
\rem^t(.,P,k): &\K^m& \to &\K^k
\end{array}$$ 
takes as input a linear form $\ell$ over $\K[X]/\langle P \rangle$,
given by means of its values $(\ell(X^i))_{0 \le i < m}$; the output
is then the values $(\ell(X^i))_{0 \le i < k}$. For $k \le m$, there
is nothing to do; for greater values of $k$, the transposed version of
the Cook-Sieveking-Kung fast Euclidean division algorithm given
in~\cite{bostan+lecerf+schost:tellegen} is given below. As for the
forward direction, for $k \ge m$, the cost of the transposed algorithm
is $O(\M(k))$.

\begin{algorithm}[H]
  \caption{$\rem^t(\ell,P,k)$}
  \begin{algorithmic}[1]
    \REQUIRE $\ell=(\ell_i)_{0 \le i < m}$, $P$ in $\K[X]$ monic of degree $m$, $k \ge m$
    \STATE $S = 1/\rev(P, m+1) \bmod X^{k-m}$
    \STATE $A = \mul^t( \sum_{0 \le i < m} \ell_{i}X^i, P, m, k-m)$
    \STATE $C = S A \bmod X^{k-m}$
    \STATE $D = \ell ~\cat~ (-\coeff(C,i))_{0 \le i < k-m}$
    \RETURN $D$
  \end{algorithmic}
\end{algorithm}

Finally, the definition of $\mulmod$ shows that for and $B$ in
$\K[X]/\langle P \rangle$, the transposed map $\mulmod^t(.,B,P)$ maps
a linear form $\ell \in (\K[X]/\langle P \rangle)^\star$ to the linear
form $B \circ \ell$ defined by
$$
\begin{array}{cccc}
B \circ \ell: &\K[X]/\langle P \rangle &\to& \K  \\
& A & \mapsto & \ell(A B).
\end{array}$$
Since we have seen that $\mulmod(.,B,P) = \rem(.,P,2m-1) \circ \mul(.,B,m-1,m)$,
we obtain the following algorithm for the transpose of modular multiplication
algorithm (see also~\cite{shoup99,bostan+lecerf+schost:tellegen});
it costs $O(\M(m))$ operations in $\K$.


\begin{algorithm}[H]
  \caption{$\mulmod^t(\ell,B,P)$}
  \begin{algorithmic}[1]
    \STATE $D = \rem^t(\ell,P,2m-1)$
    \RETURN $\mul^t(D, B, m-1, m)$
  \end{algorithmic}
\end{algorithm}

%%%%%%%%%%%%%%%%%%%%%%%%%%%%%%%%%%%%%%%%%%%%%%%%%%%%%%%%%%%%
%%%%%%%%%%%%%%%%%%%%%%%%%%%%%%%%%%%%%%%%%%%%%%%%%%%%%%%%%%%%
%%%%%%%%%%%%%%%%%%%%%%%%%%%%%%%%%%%%%%%%%%%%%%%%%%%%%%%%%%%%

\section{Trace formulas} 

Let $s$ be a positive integer and let $I$ be a maximal ideal in
$\K[X_1,\dots,X_s]$, so that $\L=\K[X_1,\dots,X_s]/I$ is a finite
extension of $\K$.  The {\em trace} $\tau_{\L/\K}$ will be denoted by
$\tau_I$; equivalently, since $\K$ is perfect, we have for $B$ in $\L$
\begin{equation}\label{eq:tr}
\tau_{I}(B)=\sum_{\x \in V} B(\x),
\end{equation}
where $V=V(I)$ lies in $\Kbar^n$. 

\paragraph{Univariate formulas.} In this paragraph, we fix $B$ in $\L$ and we 
denote by $M_{I,B} \in \K[X]$ the minimal polynomial of $B$. If we let
$V_B \subset \Kbar$ be the image of $V$ under the mapping $\x \mapsto
B(\x)$, then $M_{I,B}$ is the polynomial $\prod_{\beta \in
  V_B}(X-\beta)$.

The subfield $\K(B) \subset \L$ is isomorphic to $\K[X]/\langle
M_{I,B} \rangle$. We may then consider the trace
${\rm Tr}_{\L/\K(B)}$, which we see as an $\K[X]/\langle
M_{I,B}\rangle$-linear mapping
$$
\begin{array}{cccc}
\tau_{I,B} :& \L& \to& \K[X]/\langle M_{I,B} \rangle.
\end{array}$$
The trace ${\rm Tr}_{\L/\K(B)}$ is $n$ times the identity on $\K(B)$,
where we denote by $n$ the index $[\L:\K(B)]$. In particular, if
$A=C(B)$, for some polynomial $C$ in $\K[X]$, then $\tau_{I,B}(A)=n
\ C \bmod M_{I,B}$.

The following lemma shows how to relate $\tau_{I,B} (A)$ to the
sequence $\tau_I(A B^i)$. This is a restatement of well-known results,
see for instance~\cite{rouiller99}. In what follows, we sometimes
identify $\tau_{I,B}(A)$, which lies in $\K[X]/\langle M_{I,B} \rangle$,
to its canonical preimage in $\K[X]$.

\begin{Lemma}\label{lemma:trace:1}
  For $A$ and $B$ in $\L$, 
  $$\sum_{i \ge 0} \tau_I(A B^i) X^i = \frac{\rev( M'_{I,B}\, \tau_{I,B}
    (A) \bmod M_{I,B},m)}{\rev(M_{I,B},m+1)},$$ where
  $m=\deg(M_{I,B})$.
\end{Lemma}
\begin{proof}
First, remark that for $A$ in $\L$, $\tau_{I}(A)$ is such that for all
$\beta$ root of $M_{I,B}$ (that is, for $\beta$ in $V_B$), we have
\begin{equation}\label{eq:TrB}
\tau_{I,B}(A)(\beta) = \sum_{\x \in V,\, B(\x)=\beta} A(\x).  
\end{equation}
Then, starting from Equation~\eqref{eq:tr} and summing the geometric
series that arises, we obtain
\begin{eqnarray*}
\sum_{i \ge 0} \tau_I(A B^i) X^i &=& \sum_{\x \in V} \sum_{i \ge 0} A(\x)B(\x)^i X^i\\
&=& \sum_{\x \in V} \frac{A(\x)}{1-B(\x)X}\\
&=& \frac{\sum_{\beta \in V_B} \sum_{\x \in V,\, B(\x)=\beta} A(\x) \prod_{\beta'\ne \beta} (1-\beta' X)}
    {\prod_{\beta \in V_B}(1-\beta X)}\\
&=& \frac{\rev(N,m)}{\rev(M_{I,B},m+1)},
\end{eqnarray*}
where $N$ is the polynomial
$$\sum_{\beta \in V_B}\ \sum_{\x \in V,\, B(\x)=\beta} A(\x) \prod_{\beta'\ne \beta} (X-\beta' ).$$
In particular, for  $\beta$ in $V_B$, $N(\beta)=\sum_{\x \in V,\, B(\x)=\beta} A(\x) \prod_{\beta'\ne \beta} (X-\beta' )$,
which coincides with $M'_{I,B}(\beta)\sum_{\x \in V,\, B(\x)=\beta} A(\x)$.
Comparing with Equation~\eqref{eq:TrB}, we obtain 
$N=M'_{I,B}\, \tau_{I,B}(A) \bmod M_{I,B}$.
\end{proof}

As a first application, we consider the case where $M_{I,B}$ is known,
and we want to compute several traces of the form $\tau_I(B^i)$, for
$i=0,\dots,k-1$, for some $k \ge m$. In this case, taking $A=1$, the
previous lemma shows that the sequence $(\tau_I(B^i))_{i \ge 0}$ is
the coefficient sequence of the power series
  $$ \frac{n\, \rev(M'_{I,B} ,m)}{\rev(M_{I,B},m+1)}.$$ This leads to
the following classical algorithm.

\begin{algorithm}[H]
  \caption{TraceFromMinpoly$(M_{I,B}, n, k)$}
  \begin{algorithmic}[1]
    \REQUIRE  $M_{I,B}$ monic in $\K[X]$ of degree $m$, $k \ge 1$, $n=[\L:\K(B)]$    
    \ENSURE $(\tau_I(B^i))_{0 \le i < k}$
    \STATE\label{algo:minpolytotrace:1} $D = n\, \rev(M'_{I,B}, m)/\rev(M_{I,B}, m+1) \bmod X^k$
    \RETURN $(\coeff(D,i))_{0 \le i < k}$
  \end{algorithmic}
  \label{algo:minpolytotrace}
\end{algorithm}

\begin{Lemma}\label{lemma:computetrace}
  Algorithm~\ref{algo:minpolytotrace} correctly computes
  $(\tau_I(B^i))_{0 \le i < k}$ in time $O(\M(k))$.
\end{Lemma}

As a second application, we consider $A$ in $\K(B)$. Given traces of
the form $\tau_I(A B^i)$, we want to express $A$ as a polynomial in
$B$ (the natural context for this kind of algorithm is a situation
where we do not have access to $A$ itself but can indirectly compute
values of the form $\tau_I(A B^i)$). Various forms of this algorithm
were already known, such as Shoup's~\cite{shoup94} (which did not use
traces, but random linear forms) or Rouillier's~\cite{rouiller99}.

\begin{algorithm}[H]
  \caption{ConvertFromTrace$(t, M_{I,B}, n)$}
  \begin{algorithmic}[1]
    \REQUIRE  $t=(\tau_I(A B^i))_{0 \le i < m}$, $M_{I,B}$ monic in $\K[X]$ of degree $m$, $n=[\L:\K(B)]$
    \ENSURE $C$ in $\K[X]_m$
    \STATE $D =  1/M'_{I,B} \bmod M_{I,B}$
    \STATE $N=\rev(M_{I,B}, m+1)( \sum_{0 \le i <m} t_i X^i) \bmod X^m$
    \STATE $N^\star = \rev(N, m)$
    \STATE $C=\mulmod(N^\star, D, M_{I,B})$
    \RETURN $C/n$
  \end{algorithmic}
  \label{algo:tracetopoly}
\end{algorithm}

\begin{Lemma}
  Suppose that $n=[\L:\K(B)]$ is a unit in $\K$. If $A$ is in
  $\K(B)$, Algorithm~\ref{algo:tracetopoly} computes a polynomial $C$
  such that $A=C(B)$ in time $O(\M(m)\log(m))$.
\end{Lemma}
\begin{proof}
  Correctness follows from the remark made before
  Lemma~\ref{lemma:trace:1} that $\tau_{I,B}(A)=n \ C \bmod M_{I,B}$,
  together with the following consequence of Lemma~\ref{lemma:trace:1}:
$$ \rev( M'_{I,B}\, \tau_{I,B}  (A) \bmod M_{I,B},m) = \rev(M_{I,B},m+1) \left (\sum_{i \ge 0} \tau_I(A B^i) X^i \right )  \bmod X^m.$$
  As to the running time, the dominant part is the computation of $D$,
  which takes time $O(\M(m)\log(m))$.
\end{proof}

For latter use, we mention the transpose of algorithm
ConvertFromTrace; since the former takes as input the values of a
linear form and outputs a polynomial, the transpose will actually do
the same. As it turns out, this mapping is symmetric: this is easier
to see when considering its inverse, whose matrix in the canonical
bases is the Hankel matrix with entries $\tau_I(B^{i+j})$.

%% Remark that the non-linear part of the computation
%% (computing $D$) does not change and that all other steps are simply
%% reversed, and replaced by their transposes. The running time remains
%% $O(\M(m)\log(m))$.

%% \begin{algorithm}[H]
%%   \caption{ConvertFromTrace$^t(C,M_{I,B},n)$}
%%   \begin{algorithmic}[1]
%%    \REQUIRE $C=(c_i)_{0 \le i <m}$, $M_{I,B}$ monic in $\K[X]$ of degree $m$, $n=[\L:\K(B)]$
%%    \ENSURE  $T \in \K[X]_m$
%%     \STATE $D =  1/M'_{I,B} \bmod M_{I,B}$
%%     \STATE $(N^\star_i)_{0 \le i < m}=\mulmod^t(C/n, D, M_{I,B})$
%%     \STATE $N = (N^\star_{m-1-i})_{0 \le i < m}$
%%     \STATE $(T_i)_{0 \le i < m}=\mulmod^t(N, \rev(M_{I,B},m+1), X^m)$
%%     \RETURN $T=\sum_{0 \le i < m} T_i X^i$
%%   \end{algorithmic}
%% \end{algorithm}


\paragraph{Trace formulas: bivariate case.} 
Let $\L$ be as above. Consider $B$ and $C$ in $\L$, Below, we will
take the minimal polynomials $M_{I,B}$ and $M_{I,C}$ of respectively
$B$ and $C$ in respectively $\K[X]$ and $\K[Y]$, and we will assume
that $\K(B)$ and $\K(C)$ are linearly disjoint, so that $\K(B,C) \simeq
\K[X,Y]/\langle M_{I,B}, M_{I,C}\rangle$. Thus, the trace ${\rm
  Tr}_{\L/\K(B,C)}: \L \to \K(B,C)$
can be written as 
$$\begin{array}{cccc}
\tau_{I,B,C} :& \L& \to& \K[X,Y]/\langle M_{I,B}, M_{I,C} \rangle.
\end{array}$$
In this context, we can write a bivariate version of Lemma~\ref{lemma:trace:1}.
\begin{Lemma}
  Let $A, B, C$ be in $\L$, such that $\K(B)$ and $\K(C)$ are linearly
  disjoint, and let $M_{I,B}\in \K[X]$ and $M_{I,C}\in \K[Y]$ be the
  minimal polynomials of respectively $B$ and $C$. Then,
  $$\sum_{i,j \ge 0} \tau_I(A B^i C^j) X^i Y^j= 
  \frac{\rev( M'_{I,B}\, M'_{I,C}\, \tau_{I,B,C} (A) \bmod \langle M_{I,B}, M_{I,C}\rangle,m,n)}{\rev(M_{I,B},m+1)\, \rev(M_{I,C},n+1)},$$
  with $m=\deg(M_{I,B})$ and $n=\deg(M_{I,C})$.
\end{Lemma}
\begin{proof}
  Let us now denote by $V_{B,C}$ the image of $V$ under the mapping
  $\x \mapsto (B(\x),C(\x))$. Under the linear disjointness
  assumption, this is simply $V_B \times V_C$. As in the proof of
  Lemma~\ref{lemma:trace:1}, we now have, for $\beta$ in $V_B$ and $\gamma$ in $V_C$,
  \begin{equation}\label{eq:TrBC}
    \tau_{I,B,C}(A)(\beta,\gamma) = \sum_{\x \in V,\, B(\x)=\beta,\, C(\x)=\gamma} A(\x).  
  \end{equation}
  Still proceeding as in the previous lemma, we have
\begin{eqnarray*}
\sum_{i,j \ge 0} \tau_I(A B^i C^j) X^i Y^j &=& \sum_{\x \in V}  \frac{A(\x)}{(1-B(\x)X)(1-C(\x)Y)}\\
&=& \sum_{(\beta,\gamma) \in V_{B,C}} \frac{\sum_{\x \in V,\, B(\x)=\beta,\, C(\x)=\gamma} A(\x)}{(1-\beta X)(1-\gamma Y)}\\
&=&\frac{\sum_{(\beta,\gamma) \in V_{B,C}} \sum_{\x \in V,\, B(\x)=\beta,\, C(\x)=\gamma} A(\x)
  \prod_{\beta'\ne \beta} (1-\beta X) \prod_{\gamma' \ne \gamma} (1-\gamma Y)}
  {\rev(M_{I,B},m+1) \rev(M_{I,C},n+1) }\\
&=& \frac{\rev(N,m,n)}  {\rev(M_{I,B},m+1) \rev(M_{I,C},n+1) },
\end{eqnarray*}
with 
$$N = \sum_{(\beta,\gamma) \in V_{B,C}} \sum_{\x \in V,\, B(\x)=\beta,\, C(\x)=\gamma} A(\x)
  \prod_{\beta'\ne \beta} (X-\beta ) \prod_{\gamma' \ne \gamma} (Y-\gamma ).$$
Because of our disjointness assumption, we can rewrite $N$ as 
$$N = \sum_{\beta \in V_B} \sum_{\gamma \in V_C} \sum_{\x \in V,\, B(\x)=\beta,\, C(\x)=\gamma} A(\x)
  \prod_{\beta'\ne \beta} (X-\beta ) \prod_{\gamma' \ne \gamma} (Y-\gamma ).$$
As before, we deduce that for any $\beta$ in $V_B$ and $\gamma$ in $V_C$, 
$$N(\beta,\gamma)= M_{I,B}'(\beta)\, M_{I,C}'(\gamma) \sum_{\x \in V,\, B(\x)=\beta,\, C(\x)=\gamma} A(\x).$$
Using~\eqref{eq:TrBC}, this implies that 
$$N(\beta,\gamma)= M_{I,B}'(\beta)\, M_{I,C}'(\gamma) \tau_{I,B,C}(A)(\beta,\gamma).$$
Since this holds for all $\beta$ and $\gamma$ roots of respectively
$M_{I,B}$ and $M_{I,C}$, we deduce that
$N=M_{I,B}'\, M_{I,C}'\, \tau_{I,B,C}(A) \bmod \langle M_{I,B},\, M_{I,C}\rangle$.
\end{proof}

As in the previous paragraph, we deduce an algorithm that takes as
input traces of the form $\tau_I(A B^i C^j)$, for some $A,B,C$ in $\L$
and expresses $A$ as a polynomial in $B$ and $C$, if possible.
\begin{algorithm}[H]
  \caption{ConvertFromTrace$(t, M_{I,B}, M_{I,C}, p)$}
  \begin{algorithmic}[1]
    \REQUIRE  $t=(\tau_I(A B^iC^j))_{0 \le i < m, 0 \le i < n}$, $M_{I,B}$ monic in $\K[X]$ of degree $m$,
    $M_{I,C}$ monic in $\K[Y]$ of degree $n$, 
    $p=[\L:\K(B,C)]$
    \ENSURE a polynomial $H$ in $\K[X,Y]_{m,n}$
    \STATE $D =  1/M'_{I,B} \bmod M_{I,B}$
    \STATE $E =  1/M'_{I,C} \bmod M_{I,C}$
    \STATE $N=\rev(M_{I,B}, m+1)\rev(M_{I,C}, n+1)( \sum_{0 \le i <m, 0 \le j < n} t_{i,j} X^iY^i) \bmod \langle X^m, Y^n \rangle$
    \STATE $N^\star = \rev(N, m, n)$
    \STATE $H=\mulmod(N^\star, D E, \langle M_{I,B}, M_{I,C} \rangle)$
    \RETURN $H/p$
  \end{algorithmic}
  \label{algo:tracetopoly2}
\end{algorithm}

\begin{Lemma}
  Suppose that $p=[\L:\K(B,C)]$ is a unit in $\K$. If $A$ is in
  $\K(B,C)$, with $\K(B)$ and $\K(C)$ linearly disjoint,
  Algorithm~\ref{algo:tracetopoly2} computes a polynomial $H$ such
  that $A=H(B,C)$ in time $O(\M(m)\log(m)+\M(n)\log(n)+\M(mn))$, which
  is $O(\M(mn)\log(mn))$.
\end{Lemma}
\begin{proof}
  Correctness follows from the remark that if $A$ is in $\K(B,C)$, say
  $A=H(B,C)$, then $\tau_{I,B,C} (A)=[\L:\K(B,C)]\, H \bmod \langle
  M_{I,B}, M_{I,C} \rangle$.  As to the running time, computing
  the inverses $D$ and $E$ takes total time $O(\M(m)\log(m)+\M(n)\log(n))$;
  the subsequent products take time $O(\M(mn))$.
\end{proof}

%%%%%%%%%%%%%%%%%%%%%%%%%%%%%%%%%%%%%%%%%%%%%%%%%%%%%%%%%%%%
%%%%%%%%%%%%%%%%%%%%%%%%%%%%%%%%%%%%%%%%%%%%%%%%%%%%%%%%%%%%
%%%%%%%%%%%%%%%%%%%%%%%%%%%%%%%%%%%%%%%%%%%%%%%%%%%%%%%%%%%%

\section{Field embedding and isomorphism} 

Consider two irreducible polynomials $P$ and $Q$ in respectively
$\K[X]$ and $\K[Y]$, with respective degrees $m$ and $n$. We assume
that $\gcd(m,n)=1$, and that $\K$ is a finite field. As a result, the
ideal $\langle P, Q\rangle$ is maximal in $\K[X,Y]$, and $XY$
generates the field $\L=\K[X,Y]/\langle P, Q \rangle$.

Let $R \in \K[X]$ be the minimal polynomial of $XY$ in the extension
$\K\to \L$; then, $R$ is irreducible of degree $m n$ and we have embeddings
of the form
$$\begin{array}{cccc}
\varphi_X: & \K[X]/\langle P \rangle & \to & \K[Z]/\langle R \rangle\\
& X & \mapsto & S
\end{array}$$
and
$$\begin{array}{cccc}
\varphi_Y: & \K[Y]/\langle Q \rangle & \to & \K[Z]/\langle R \rangle\\
& Y & \mapsto & T,
\end{array}$$
for some polynomials $S$ and $T$ in $\K[Z]$ of degree less than $mn$; 
we also have an isomorphism of the form
$$\begin{array}{cccc} 
\Phi:&  \K[X,Y]/\langle P,Q\rangle & \to & \K[Z]/\langle R \rangle \\
&  X & \mapsto & S \\
&  Y & \mapsto & T \\
&  XY & \mapsfrom & Z.
\end{array}$$

In what follows, we discuss algorithms for computing $R$, then
applying $\varphi_X$ and $\varphi_Y$ as well as their inverses (when
well-defined), as well as $\Phi$ and its inverse. Except from the
computation of $R$, these are all linear algebra problems. 

If $R,S,T$ are known, then a direct solution is available: modular
composition, which (for instance to compute $\varphi_X$) amounts to
computing $\varphi_X(F)$ as $F(S) \bmod R$. The algorithms given below
are better than such a direct approach.

In what follows, we write $\tau_P,\tau_Q,\tau_R$ for the traces modulo
the ideals $\langle P\rangle\subset \K[X]$, $\langle Q \rangle \subset
\K[Y]$ and $\langle R \rangle \subset \K[Z]$. Let as well $I$ be the ideal
$\langle P, Q\rangle$ in $\K[X,Y]$.

Note finally that algorithms TraceFromMinpoly and ConvertFromTrace
take as inputs some indices (denoted by $n$ and $p$ in their
pseudo-code) that will always equal 1 here.

%% In view of Lemma~\ref{lemma:computetrace}, for $k \ge 1$, we can
%% compute $(\tau_P(X^i))_{0 \le i < k}$ and $(\tau_Q(Y^i))_{0 \le i <
%%   k}$ in $O(\M(k))$ operations.

%%%%%%%%%%%%%%%%%%%%%%%%%%%%%%%%%%%%%%%%%%%%%%%%%%%%%%%%%%%%

\subsection{Computing $R$} 

The algorithm for computing $R$ is well-known (see for
instance~\cite{BoFlSaSc06}), and we just recall it here for
completeness. It is based on the following lemma.

\begin{Lemma}
  For $i \ge 0$, $\tau_R(Z^i) = \tau_P(X^i) \ \tau_Q(Y^i)$.
\end{Lemma}
\begin{proof}
  Because $\Phi$ is an isomorphism, $\tau_R(Z^i) = \tau_I(X^i
  Y^i)$. Because $\K[X,Y]/I$ is the tensor product of $\K[X]/P$ and
  $\K[Y]/Q$, $\tau_I(X^i Y^i)=\tau_P(X^i) \ \tau_Q(Y^i)$.
\end{proof}

The algorithm follows: we first compute the traces of the powers $X^i$
and $Y^i$, deduce those of $Z^i$, and finally recover the minimal
polynomial of $Z$ from them. To do the latter, one may use Newton
iteration (as in~\cite{Schoenhage82}) if the characteristics of $\K$
is greater than $mn$; in general, we use the Berlekamp Massey
algorithm (or rather a fast variant thereof), which is slightly slower
($O(\M(mn)\log(mn))$ instead of $O(\M(mn))$). In any case, 
this is quasi-linear (previous algorithms were at least quadratic,
see the discussion in~\cite{BoFlSaSc06}).

\begin{algorithm}[H]
  \caption{ComputeR$(P, Q)$}
  \begin{algorithmic}[1]
    \REQUIRE $P$ monic of degree $m$ in $\K[X]$, $Q$ monic of degree $n$ in $\K[Y]$
    \ENSURE $R$ monic of degree $mn$ in $\K[Z]$
    \STATE  $t$=TraceFromMinpoly$(P,2mn,1)$
    \STATE  $u$=TraceFromMinpoly$(Q,2mn,1)$
    \STATE $v=(t_i u_i)_{0 \le i < 2mn}$
    \RETURN BerlekampMassey$(v)$
  \end{algorithmic}
  \label{algo:R}
\end{algorithm}

\begin{Lemma}
  Algorithm~\ref{algo:R} returns $R$ using $O(\M(mn)\log(mn))$ operations in
  $\K$.
\end{Lemma}
\begin{proof}
  Correctness is straightforward.  Lemma~\ref{lemma:computetrace}
  shows that computing the traces of the powers $X^i$ and $Y^i$ takes
  time $O(\M(mn))$; deducing those of the powers $Z^i$ takes $O(mn)$
  operations, and the Berlekamp Massey algorithm costs
  $O(\M(mn)\log(mn))$ operations.
\end{proof}


%%%%%%%%%%%%%%%%%%%%%%%%%%%%%%%%%%%%%%%%%%%%%%%%%%%%%%%%%%%%

\subsection{Embedding} 

Next, we show how to compute an embedding, say $\varphi_X$. Let thus
$F$ and $G$ be in respectively $\K[X]/\langle P \rangle$ and
$\K[Z]/\langle R \rangle$, such that $G=\varphi_X(F)$. Assuming
$S=\varphi_X(X)$ is known, using Brent and Kung's modular composition
techniques~\cite{brent+kung} to compute $G$ as $G=F(S) \bmod R$, this
can be done in $O(n m^{(\omega+1)/2})$ operations in $\K$, since we
evaluate a polynomial of degree $m$ modulo the polynomial $R$ of
degree $mn$ (see the analysis in~\cite{shoup94}).

In this section, we show how to do this in quasi-linear time in $mn$,
together with the inverse operation. We use the following lemma, which
generalizes the result used to compute $R$.

\begin{Lemma}\label{lemma:traces:PQR}
  For $i \ge 0$, $\tau_R(G Z^i) = \tau_P(F X^i) \ \tau_Q(Y^i)$ and
  $\tau_R(G S^i) = n\ \tau_P(F X^i)$.
\end{Lemma}
\begin{proof}
As before: $\tau_R(G Z^i) = \tau_I(F X^i Y^i)$ proves the first
equality and $\tau_R(G S^i) = \tau_I(F X^i)$ proves the second one.
\end{proof}

Given $F$, we first show how to compute $G$. To do this, we compute
$\tau_R(G Z^i)$ for $i=0,\dots,mn-1$, by means of the previous lemma,
and apply Algorithm~\ref{algo:tracetopoly} (ConvertFromTrace); this
leads to the following algorithm. Remark that the values $\tau_P(F
X^i)_{0 \le i < mn}$ are simply the values of $F \circ \tau_P$ at
$(X^i)_{0 \le i < mn}$; we compute them by computing $F \circ \tau_P$
(that is, on the first $m$ powers of $X$), then applying $\rem^t$ to
obtain all following values.

\begin{algorithm}[H]
  \caption{Embed$(F,P,R)$}
  \begin{algorithmic}[1]
    \STATE $t$=TraceFromMinpoly$(P,m,1)$
    \STATE $u$=TraceFromMinpoly$(Q,mn,1)$
    \STATE\label{algo:embed:2} $\ell = \mulmod^t(t,F,P)$
    \STATE $(\ell^\star_i)_{0 \le i < mn} = \rem^t(\ell, P, mn)$
    \STATE $v=(\ell^\star_i u_i)_{0 \le i < mn}$
    \RETURN ConvertFromTrace$(v, R, 1)$
  \end{algorithmic}
  \label{algo:embed}
\end{algorithm}

The running time analysis of this algorithm is straightforward, the most
costly step being the call to ConvertFromTrace; we record this result in 
the following lemma. 

\begin{Lemma}\label{lemma:algo:embed}
  Algorithm~\ref{algo:embed} correctly computes $\varphi_X(F)$
  using $O(\M(mn)\log(mn))$ operations in~$\K$.
\end{Lemma}



For the inverse, we take $G$ in $\K[Z]/\langle R \rangle$ of the form
$G=\varphi_X(F)$, and compute $F$. As above, will achieve this by
computing $\tau_P(F X^i)$ and recovering $F$. Using the first equality
of Lemma~\ref{lemma:traces:PQR} in the form $\tau_P(F X^i) =\tau_R(G
Z^i)/\tau_Q(Y^i)$ would lead to a simple algorithm, but some traces
$\tau_Q(Y^i)$ may vanish. Instead, we use the second equality, which
we rewrite as $\tau_P(F X^i) =\tau_R(G S^i)/n$.

In order to find the traces $\tau_R(G S^i)$, we use transposition
techniques. We mentioned above that applying $\varphi_X$ amounts to a
modular composition: given $F$, $G$ is given by $G= F(S) \bmod R$. As
a consequence, the transposed map $\varphi_X^t$ computes the values
$\ell \mapsto (\ell(S^i))_{0 \le i < m}$, for $\ell$ in
$(\K[Z]/\langle R \rangle)^\star$ (this duality between modular
composition and power projection was observed in~\cite{shoup94}).

The following algorithm for $\varphi_X^t$ is the transpose of
Algorithm~\ref{algo:embed}: the computations of $t$ and $u$ are
unchanged (they do not involve the input with respect to which we do
the transposition) and the other steps are all transposed and their
order is reversed. The only nontrivial ingredient is the transpose of
Step~\ref{algo:embed:2} of that algorithm, which originally computes
$\ell= \mulmod^t(t,F,P)$ from $F$; unrolling the definition shows that
this map is self-dual, so this step remains unchanged.

\begin{algorithm}[H]
  \caption{Embed$^t(\ell,P,R)$}
  \begin{algorithmic}[1]
  \STATE $t$=TraceFromMinpoly$(P,m,1)$
  \STATE $u$=TraceFromMinpoly$(Q,mn,1)$
  \STATE $v$=ConvertFromTrace$(\ell, R, 1)$
  \STATE $\ell^\star=\sum_{0 \le i < mn} v_i u_i X^i$
  \STATE $G = \ell^\star \bmod P$
  \RETURN $\mulmod^t(t,G,P)$
  \end{algorithmic}\label{algo:embedT}
\end{algorithm}

Having an algorithm for $\varphi_X^t$, we deduce an algorithm to
compute $F$ from $G=\varphi_X(F)$. Correctness follows from the fact
mentioned above that $\tau_P(F X^i) =\tau_R(G S^i)/n$ for all $i$.

\begin{algorithm}[H]
  \caption{InverseEmbed$(G,P,R)$}
  \begin{algorithmic}[1]
  \STATE $t$=TraceFromMinpoly$(R,mn,1)$
  \STATE $\ell = \mulmod^t(t,G,R)$
  \STATE $v={\rm Embed}^t(\ell,P,R)$
  \RETURN ConvertFromTrace$(v, R, 1)/n$
  \end{algorithmic}\label{algo:inverseEmbed}
\end{algorithm}

\begin{Lemma}
  Given $G$ in the image of $\varphi_X$, Algorithm~\ref{algo:inverseEmbed}
  correctly computes $F$ such that $G=\varphi_X(F)$ using
  $O(\M(mn)\log(mn))$ operations in~$\K$.
\end{Lemma}
\begin{proof}
  The cost of Algorithm~\ref{algo:embedT} remains $O(\M(mn)\log(mn))$,
  and all extra steps using in Algorithm~\ref{algo:inverseEmbed} 
  cost $O(\M(mn)\log(mn))$ as well.
\end{proof}


%%%%%%%%%%%%%%%%%%%%%%%%%%%%%%%%%%%%%%%%%%%%%%%%%%%%%%%%%%%%

\subsection{Isomorphism} 

In all that follows, without loss of generality, {\em we assume that
  $m\le n$}. We are not able to give an algorithm for the isomorphism
$\Phi$ that would be as efficient as those for embedding. Instead, we
provide two such algorithms, with different domains of applicability.
In a second time, we discuss computing the inverse of $\Phi$,
for which we use techniques similar to those use for InverseEmbed.

\paragraph{First case: $m$ is small.}
We start by a direct application of the results in the previous
subsection, which is well-suited to situations where $m$ is small compared to $n$. 

Let $F$ be in $\K[X,Y]/\langle P,Q\rangle$ and let
$G=\Phi(F)$. Writing $F=\sum_{0 \le i < m} F_i X^i$, with all $F_i$ in
$\K[Y]/\langle Q \rangle$, we obtain the following straightforward
algorithm to compute $G$: compute all $\varphi_Y(F_i)$, and use a
Horner scheme to deduce $G$ as $G=\sum_{0 \le i < m} \varphi_Y(F_i) S^i$.
\begin{algorithm}[H]
  \caption{ChangeBasis1$(F,P,Q,R)$}
  \begin{algorithmic}[1]
    \STATE $S={\rm Embed}(X,P,R)$
    \STATE $G=0$
    \FOR{$i=m-1,\dots,0$}
    \STATE $G_i = {\rm Embed}(\coeff(F,X,i),Q,R)$
    \STATE $G = GS+G_i \bmod R$
    \ENDFOR
    \RETURN $G$
  \end{algorithmic}
  \label{algo:iso1}
\end{algorithm}

\begin{Lemma}
  Algorithm~\ref{algo:iso1} correctly computes $\Phi(F)$ using
  $O(m\M(mn)\log(mn))$ operations in~$\K$.
\end{Lemma}

The proof is a direct application of Lemma~\ref{lemma:algo:embed}.  As
was the case for $\varphi_X$, we will need the transpose of this
algorithm as well (todo: more explanations)
\begin{algorithm}[H]
  \caption{ChangeBasis1$^t(\gamma,P,Q,R)$}
  \begin{algorithmic}[1]
    \STATE $S={\rm Embed}(X,P,R)$
    \STATE $\ell=0$
    \FOR{$i=0,\dots,m-1$}
    \STATE $\ell_{i,j} = ({\rm Embed}^t(\gamma,Q,R))_j$, {\bf for} $j=0,\dots,n-1$
    \STATE $\gamma = \mulmod^t(\gamma,S,R)$
    \ENDFOR
    \RETURN $\ell$
  \end{algorithmic}
  \label{algo:tiso1}
\end{algorithm}

\paragraph{Second case: $m$ not small.}
The previous algorithm is most efficient when $m$ is small; now, we
propose an alternative solution that does better when $m$ and $n$ are
of the same order of magnitude. 

This approach is based on baby steps / giant steps techniques, as in
Brent and Kung's modular composition algorithm, but uses the fact that
$Z=\Phi(XY)$ to reduce the cost. Given $F$ in $\K[X,Y]/\langle
P,Q\rangle$, let us write
\begin{eqnarray*}
F&=&\sum_{i=0}^{m-1}\sum_{j=0}^{n-1} f_{i,j}X^i Y^j\\
&=&\sum_{i=0}^{m-1}\sum_{j=0}^{n-1} f_{i,j}X^i Y^i Y^{j-i}\\
&=&\sum_{k=-m+1}^{n-1}\sum_{i=0}^{m-1} f_{i,i+k}(XY)^i Y^k\\
&=&\frac{1}{Y^{m-1}} \sum_{k=0}^{m+n-2} H_k(XY) Y^k,
\end{eqnarray*}
with $H_k(Z)=\sum_{0 \le i < m-1} f_{i,i+k-m+1} Z^i$ for all $k$.
This implies that $G=\Phi(F)$ has the form
$$G = \frac{1}{T^{m-1}}\widetilde{G} \mod R\quad\text{with}\quad
\widetilde{G}=\sum_{k=0}^{m+n-2} H_k T^k.$$ We use baby steps / giant
steps techniques from~\cite{LeMeSc13} (inspired by Brent and Kung's
algorithm) to compute $G$, reducing the problem to polynomial matrix
multiplication. Let
$$n'=m+n-1,\quad p=\lceil \sqrt {n'} \rceil,\quad q=\lceil
n'/p\rceil,$$ so that $n \le n' \le 2n-1$ and $p\simeq q \simeq
\sqrt{n}$.  For baby steps, we compute the polynomials $T_i=T^i \bmod
R$, which have degree at most $mn-1$; we write $T_i = \sum_{0 \le j <
  n} T'_{i,j} Z^{jm}$, with $T'_{i,j}$ of degree less than $m$, and
build the polynomial matrix $M_{T'}$ with entries $T'_{i,j}$.  We also
define the matrix $M_H=[H_{iq+j}]_{0 \le i <p, 0 \le j < q}$
containing the polynomials $H_k$ organized in a row-major fashion, and
compute the product $M_V=M_H M_T$. We can then recompose polynomials
from the rows of $M_V$, and conclude with giant steps, using Horner's
scheme to obtain $G$.
\begin{algorithm}[H]
  \caption{ChangeBasis2$(F,P,Q,R)$}
  \begin{algorithmic}[1]
    \STATE $n'=m+n-1$, $p=\lceil \sqrt {n'} \rceil$, $q=\lceil n'/p\rceil$
    \STATE\label{iso2:2} $T={\rm Embed}(Y,Q,R)$
    \STATE\label{iso2:3} $U=1/T \bmod R$
    \STATE\label{iso2:4} $T'=[T^i \bmod R]_{0 \le i < q}$
    \STATE $M_{T'}=[T'_{i,j}]_{0\le i < q, 0, \le j < n}$ \hfill $T'_{i,j}$ as defined above
    \STATE $M_H=[H_{iq+j}]_{0 \le i <p, 0 \le j < q}$ \hfill $H_k$ as defined above
    \STATE\label{iso2:7} $M_V = M_H M_{T'}$
    \STATE $V=[\sum_{0 \le j <n} {M_V}_{i,j} Z^{jm} ]_{0 \le i <p}$
    \STATE $V^\star=[V_i \bmod R]_{0 \le i <p}$
    \STATE $G=0$
    \FOR{$i=p-1,\dots,0$}\label{iso2:11}
    \STATE $G=T^qG+V^\star_i \bmod R$
    \ENDFOR
    \STATE\label{iso2:14} $G=G U^{m-1} \bmod R$
    \RETURN $G$
  \end{algorithmic}
  \label{algo:iso2}
\end{algorithm}

\begin{Lemma}
  Algorithm~\ref{algo:iso2} correctly computes $\Phi(F)$ using
  $O(\M(m) n^{(\omega+1)/2} )$ operations in~$\K$.
\end{Lemma}
\begin{proof}
  Remark first that $n'=O(n)$, and that $p$ and $q$ are both
  $O(\sqrt{n})$. Steps~\ref{iso2:2},~\ref{iso2:3} and~\ref{iso2:14}
  cost $O(\M(mn)\log(mn))$ operations. Steps~\ref{iso2:4} (the baby
  steps) and the loop at Step~\ref{iso2:11} (the giant steps) cost
  $O(\sqrt{n}\M(mn))$. The dominant cost is the matrix product at
  Step~\ref{iso2:7}, which involves matrices of size $O(\sqrt{n})
  \times O(\sqrt{n})$ and $O(\sqrt{n}) \times O(n)$, with polynomial
  entries of degree $m$: this takes $O(\M(m) n^{(\omega+1)/2})$ 
  operations in $\K$.
\end{proof}

As before, we will need the transpose of this algorithm (todo: more
explanations).
\begin{algorithm}[H]
  \caption{ChangeBasis2$^t(\gamma,P,Q,R)$}
  \begin{algorithmic}[1]
    \STATE $n'=m+n-1$, $p=\lceil \sqrt {n'} \rceil$, $q=\lceil n'/p\rceil$
    \STATE $T={\rm Embed}(Y,Q,R)$
    \STATE $U=1/T \bmod R$
    \STATE $T'=[T^i \bmod R]_{0 \le i < q}$
    \STATE $M_{T'}=[T'_{i,j}]_{0\le i < q, 0, \le j < n}$ \hfill $T'_{i,j}$ as defined above
    \STATE $\gamma = \mulmod^t(\gamma, U^{m-1}, R)$
    \STATE $V^\star=[\ ]_{0 \le i < p}$
    \FOR{$i=0,\dots,p-1$}
    \STATE $V^\star_i = \gamma$
    \STATE $\gamma = \mulmod^t(\gamma,T^q,R)$
    \ENDFOR
    \STATE $V = [\rem^t(V^\star_i,R,mn+m-1)]_{0 \le i < p}$
    \STATE $M_V = [(V_{i})_{jm,\dots,jm+2m-1}]_{0 \le i < p, 0 \le j < n}$
    \STATE $M_H = \mul^t(M_V, M_{T'})$
    \STATE $H=[{M_H}_{0,0},\dots,{M_H}_{0,q-1},\dots,{M_H}_{p-1,q-1}]$
    \STATE $\ell=[\coeff(H_{i-j+m-1},i)]_{0 \le i < m, 0 \le j < n}$
    \RETURN $\ell$
  \end{algorithmic}
  \label{algo:tiso2}
\end{algorithm}

\paragraph{Inverse isomorphism.} Finally, as in the previous subsection, we 
give an algorithm for the inverse of $\Phi$, by using the transpose of
$\Phi$. Let $G$ be in $K[Z]/\langle R\rangle$, and let $F$ be such
that $G=\Phi(F)$. In order to compute $F$, we are going to compute the
traces $(\tau_I(F X^i Y^j))_{0 \le i < m, 0 \le i < n}$, since we can
then use them to recover $F$ using the bivariate version of
ConvertFromTrace.

Through the isomorphism $\Phi$, we see that the traces we need are
$(\tau_R(G S^i T^j))_{0 \le i < m, 0 \le i < n}$, which can be
rewritten as $(\ell(S^i T^j))_{0 \le i < m, 0 \le i < n}$, where $\ell
= G \circ \tau_R$. Now, as in the univariate case, the operation $\ell
\mapsto (\ell(S^i T^j))_{0 \le i < m, 0 \le i < n}$ is the dual of the
modular composition $F \in \K[X,Y]/I \mapsto F(S,T) \bmod R$, that is,
the dual of $\Phi$. Thus, given an algorithm ChangeBasis, the inverse
algorithm is as follows:


\begin{algorithm}[H]
  \caption{InverseChangeBasis$(G,P,Q,R)$}
  \begin{algorithmic}[1]
   \STATE $t = {\rm TraceFromMinpoly}(R, mn, 1)$
   \STATE $\ell =\mulmod^t(t, G, R)$
   \STATE $v = {\rm ChangeBasis}^t(\ell, P, Q, R)$
   \RETURN ${\rm ConvertFromTrace}(v, P, Q, 1)$
  \end{algorithmic}
  \label{algo:i_iso1}
\end{algorithm}

The cost of the tranpose algorithm ${\rm ChangeBasis}^t$ is the same
as that of the direct one, up to $O(mn)$. Thus, for both solutions
that we give to compute ChangeBasis, the cost of Step 3 in
Algorithm~\ref{algo:i_iso1} is dominant, so that the inverse algorithm
has the same asymptotic cost as the direct one.

\paragraph{Summary.} For $m \le n$, and neglecting logarithmic factors,
our two solutions for $\Phi$ (or its inverse) have respective costs
$O\tilde{~}(m^2 n)$ and $O\tilde{~}(m n^{(\omega+1)/2})$. Writing
$\delta=mn$, the minimum of the two is
$O\tilde{~}(\delta^{2\omega/(\omega+1)})$.



\bibliographystyle{plain} \bibliography{defeo}

\end{document}




